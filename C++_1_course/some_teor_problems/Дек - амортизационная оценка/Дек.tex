\documentclass[12pt]{article}
\usepackage[utf8]{inputenc}
\usepackage[russian]{babel}
\usepackage{soul}
\usepackage{amsmath}
\usepackage{amssymb}
\title{Дэк}
\date{19 октября 2016}
\author{\copyright~~И. И. Гридасов}
 
\begin{document}
\maketitle
Реализация структуры Дек:

Храним буфер(массив фиксированной длины), в который будем записывать элементы. Будем хранить его размер \textit{size}.
Также будем хранить два индекса(или указателя), которые указывают на первый и последний элемент в массиве.
Если количество элементов стало больше, чем размер буфера, то создадим буфер в 2 раза большей длины и перекопируем
элементы из старого буфера в новый, старый буфер удалим. 
В процессе жизни Дека будем поддерживать инвариант: Если пройти по элементам с левого указателя до правого(считая, что 
с последнего элемента буфера мы переходим в первый), мы получим все элементы Дека в нужном порядке, начиная с первого и
заканчивая последним. Тогда за $O(n)$ мы можем пройтись по элементам Дека в правильном порядке(с первого до последнего).

Описание операций:

$Resize$: Когда, кол-во элементов, которые нужно хранить превысило размер буфера, то нужно создавать новый, больший буфер.
Например, создадим массив в 2 раза большей длины, пройдёмся по элементам Дека и перекопируем их в новый буфер, установив
левый указатель в начало нового буфера, а правый на последний записанный элемент. Время $O(n)$ в него входит создание буфера
из $2n$ элементов и перекопирование $n$ элементов в новый буфер.

$push\_back$: увеличиваем указатель на последний элемент на 1 и записывем новое число в эту ячейку, 
в случае если вышли за границу массива(за правую), то перемещаем указатель в начало массива и записываем новое
число в данную ячейку. Если же правый указатель ещё до операции стоял там же, где и первый, то значит весь буфер
уже заполнен и нужно запустить Resize и уже после добавить элемент, выше указанным способом.
Не забыть сдлеать \textit{size++}

$pop\_back$: просто уменьшаем правый указатель на 1, при переходе через левую границу переходим в последний элемент.
\textit{size -= 1}. Время работы $O(1)$

Аналогично выполняются операции $push\_front$ и $pop\_front$, при добавлении левый указатель сдвигается влево, а при удалении вправо.

$operator[]:$ Пусть $arr$ - буфер, $left, size$ - левый индекс и размер буфера соответственно. Тогда $i-ый$ элемент находится в ячейке
$arr[(left + i) mod size]$. время $O(1)$

\textbf{Доказательство Асимптотики.} Покажем, что амортизационное время работы $push\_back$ и $push\_front$ $O(1)$.

Метод Бухгалтерского учёта: Повесим на каждое число по 3 монетки. Пусть на непосредственное добавление элемента хватает одной монетки.
Далее каждый элемент либо может быть удален, тогда ещё одной монетки хватит на обеспечение операции $pop\_back$ и $pop\_front$,
Так как они выполняются за $O(1)$.
Итого потратили только 2 из 3. 
Либо может дожить до перехода в новый буфер, но тогда заметим, что успело накопиться хотя бы $2 * (\frac{size\_buff}{2}) = size\_buff = size = n$ монет, которые не удалялись,
Так как при создании буфера в нём было лишь $\frac{size\_buff}{2}$ элементов. 
А значит, с новодобавленных элементов можно собрать монетки на операцию $Resize$. 
Так как эта операция выполняется за $O(n)$, то можно выбрать монетку такой, чтобы $n$ монет хватило на эту операцию.

Метод потенциалов: В методе потенциалов, главную роль играет нахождение функции от состояния, такой, что $\Phi(D_0) = 0$ и $\Phi(D_i) \geqslant 0$.
В данном примере, как и в динамическом массиве, подойдёт $\Phi(D_i)$ равная, кол-ву элементов в структуре. 
Тогда, если $\Delta\Phi(i) = \Phi(i) - \Phi(i - 1)$, то амотризационная стоимость вычисляется как $a_i = t_i + \Delta\Phi(i)$.
 Где $t_i$ - непосредственно время, необходимое для выполнения операции. 
 В нашем случае для операции $Resize$ можно считать $t_i = n$, где $n$ - кол-во элементов в Деке, так как асимптотика данной операции $O(n)$.
 А для других операций $t_i = 1$(без учёта возможного пересоздания буфера).
 Можно взять $\Phi(i) = 2 \frac{size * size}{size\_buff}$. 
 Тогда для операций удаления:
\begin{align*}
  	a_i = t_i + 2 * \left(\frac{(size - 1) * (size - 1)}{size\_buff} - \frac{size * size}{size\_buff}\right) \leqslant t_i = 1.
\end{align*}
Для операции добавления элемента без $Resize:$
 \begin{align*}
	a_i = t _i + 2 * \left(\frac{(size + 1) * (size + 1)}{size\_buff} - \frac{size * size}{size\_buff}\right) = \\
	t_i + 2 * \left(\frac{2 * size + 1}{size\_buff}\right) \leqslant t_i + 2 * \frac{2 * (size +1)}{size\_buff} \\
	\leqslant t_i + 2 * 2 = t_i + 4 = 5
\end{align*}
А для операций добавления c Resize: 
 \begin{align*}
	a_i = t _i + 2 * \left(\frac{(size + 1) * (size + 1)}{size\_buff\_new} - \frac{size * size}{size\_buff}\right) = \\
	t_i + 2 * \left(\frac{(size + 1) * (size + 1)}{2 * size\_buff} - \frac{size * size}{size\_buff}\right) = 
	t_i + \frac{-size * size + 2 * size + 1}{size\_buff} = \\
	t_i - size + 2 + \frac{1}{size} \leqslant t_i - size + 3 = size - size + 3 = 3	
\end{align*}
Таким образом все $a_i \leqslant 5$, А значит амортизационное время работы каждой операции $O(1)$, что и требовалось показать.
\end{document}