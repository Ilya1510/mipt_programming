\documentclass[12pt]{article}
\usepackage[utf8]{inputenc}
\usepackage[russian]{babel}
\usepackage{soul}
\usepackage{amsmath}
\usepackage{amssymb}
\title{Интересная структура}
\date{22 октября 2016, 19 декабря 2016}
\author{\copyright~~И. И. Гридасов}
 
\begin{document}
\maketitle

Чтобы избавиться от недостатков предыдущей версии, при запросе удаления элемента не будем удалять его, 
а будем просто помечать, что такой элемент удалён. При этом операция добавления останется той же, 
и по уже доказанному, её амортизационная оценка $O(\log n)$. Но так как могут быть элементы одинкаково значения,
то чтобы асимптотика поиска не ухудшилась, будем поддерживать инвариант: в группе среди элементов одинакового
значения, сначала идут неудалённые, а потом удалённые. 

Тогда операция удаления примет следующий вид: Сначала находим группу, в которой находится элемент, который надо удалить. Это делается прямым проходом по всем группа за $O(\log n)$. Теперь в этой группе бин-поиском ищем самый правый неудалённый элемент с таким значением. Помечаем его как удалённый.

Перестройка: Так как мы не удаляем элементы, а помечаем, то фактически элементов, которые занимают место в структуре, больше, чем число неудалённых элементов. Для того, чтобы кол-во помеченных элементов в структуре было
не слишком велико и чтобы не испортилась асимптотика операций, будем как-только кол-во помеченных элементов стало равно кол-ву непомеченных, то делаем перестройку.
Тогда перед каждой перестройкой было сделано хотя бы $n$, (где $2n$ - сейчас всего элементов в структуре) операций удаления. Так как перестройка делается за $O(n\log n)$, то 
мы можем придумать монетки, так чтобы с каждой операции удаления мы брали по $\log n$ монеток. Функция поиска также работает за нужную асимптотику, так как кол-во групп осталось
$O(\log n)$. Операция добавления по предыдущим док-ствам осталась работать за $O(\log n)$.

Общее представление структуры:

Храним переменную size или n - кол-во элементов в структуре.
Будем хранить список указателей на начало каждого набора в порядке увеличения степени двойки.
При этом буфер каждого набора пусть будет в 2 раза больше, чем кол-во элементов, которое должен хранить.
Таким образом для числа 19, будем хранить список из 5 указателей на буферы, размером 2, 4, 8, 16, 32 соотвественно.
В первом будет лежать 1 элемент, во втором 2 и в последнем 16 элементов,
все они лежат в первых ячейках соответствующих буферов. Оставшиеся буферы пока пустуют(но они есть).
При таком представлении структуры, кол-во занимаемой памяти $\leqslant 4n$,
так как будем поддерживать инвариант, что последний буфер заполнен, а значит так как суммарно 
все буферы без последнего занимают на 2 ячейки меньше, чем один последний буфер,
то суммарная длина всех буферов $\leqslant$ 2 * длина последнего $=$ 4 * кол-во эл-тов в последнем буфере
$\leqslant$ 4 * кол-во всех эл-тов(занятых ячеек) во всех буферах $= 4n$.

Функция Search(x):

Перебираем все наборы массивов, в каждом из них запускаем бин-поиск.
Кол-во наборов $O(\log n)$, длина каждого набора $O(n)$, следовательно бин-поиск
отработает за время $O(\log n)$.
Итоговая асимптотика операции: $O(\log^2 n)$.

Функция Insert(x):

Пусть первый буфер пуст, то есть size делится на 2. 
(Проверку на пустоту далее можно делать поддерживая переменную $tmp = size$, 
а при переходе к следующей итерации делать $tmp /= 2$).
Тогда просто поставим этот элемент на первое место в первом буфере.
Если же нет, то сольём его в первый буфер, так как буфер на 2 эл-та так можно сделать. 
Теперь если второй буфер пуст, то просто перекопируем эти два элемента туда.
Иначе проделаем тоже самое, сольём эти 2 элемента с элементами из второго буфера во второй буфер.
Слияние делаем, так чтобы второй буфер заполнялся с конца, 
это позволит нам не создавая новой памяти просто прилить один буфер к другому.
Если же мы такими действиями дошли до последнего буфера, то есть теперь он заполнен не на половину как должен а полностью, 
то создаем новый буфер размером в 2 раза больше этого буфера, перекопируем все элементы в него. добавляем указатель на него
в конец списка указателей на буферы.
Операция выполняется за $O(2^{i})$, где $i$ - кол-во единиц в конце числа n в его двоичном представлении.
$2^1+\ldots+2^{i+1} + 2^{i+1} = 2^{i+2} +^{i+1} - 1 = O(2^i)$,  
где $2^{i}$ тратится на слияние полного $i-1-ого$ буфера, заполненного полностью,
и $i-ого$, заполненного ''нормально'', то есть на половину, в каждом из них по $2^{i-1}$ эл-ов. 
И последнее $2^{i+1}$ - перенос эл-ов из заполненного буфера в первую половину следующего.
И ещё может потребоваться создать новый буфер размером $2^{i+2}$, но это также делается за $O(2^{i})$.
Увеличиваем $size$ на 1.

Функция Delete(x):

За $O(\log n)$ найдём какому набору принадлежит эл-т, который надо удалить.
Если нет непустых буферов, меньших, чем тот, в котором надо удалить, то просто раскидываем его элементы, 
во все меньшие буферы по порядку, пропустив элемент который надо удалить. 
Это мы сделали за $O(2^{i})$, где $i$ - кол-во нулей в конце числа в двоичном представлении числа n.
Так же так как теперь последний буфер пуст, то освободим память, которую он занимал.
Так как её размер $2^{i+1}$, то её удаление выполняется за $O(2^{i})$.
Если же наш буфер не меньший, то возьмём самый маленький непустой буфер и один его эл-т запишем на место эл-та, который надо удалить.
А оставшиеся раскидаем на меньшие действия вышеописанным алгоритмом. Осталось бин-поиском найти место для эл-та, который мы
поставили на место удаляемого. Асимптотика бин-поиска $O(\log n)$. Асимптотика всей операции $O(\log n + 2^{i})$, где $i$ - кол-во
нулей в конце числа в двоичном представлении числа $n$. 
Уменьшаем $size$ на 1.

Докажем, что учётное время работы каждой операции $O(\log n)$:

Воспользуемся методом бухгалтерского учёта для операции Insert. 
Пусть при поступлении новой монетки она приносит кол-во монеток равное числу существующих буферов(включая пустые) + 1.
Это число есть $O(\log n)$ благодаря тому, что последний буфер точно заполнен эл-тами.
Мы кладём на каждый буфер по монетке и 1 остаётся.
Тогда при оплате за новый запрос, возьмём монетки с последней единицы, которую мы затронули при запросе.
Если этой единицы не было то есть $size$ делится на 2(младший бит в двоичном представлении равен 0), то потратим отложенную монетку. 
Иначе возьмём монетки с самой правой единицы, которая участвовала в операции, пусть её номер $i$(начиная с 1).
На ней накопилось $2^{i}$ монет, первая пришла с числа вида $100...0(i)...$, а предпоследняя с $111...1(i)...$. 
Значит этих монет хватит на совершение операции, так как её асимптотика $O(2^{i})$.
Отсюда получаем, что амортизированная стоимость операции $O(\log n)$

Аналогичное док-ство можно провести для операции Delete.
Найдём, что амортизированная оценка $O(2^{i})$, 
где $i$ - кол-во нулей с конца в двоичном представлении числа $n$ равна $O(\log n)$.
Теперь добавляем в каждую операцию ещё $O(\log n)$(бин-поиск и поиск нужной группы),
получаем, что амортизированная стоимость каждой операции $O(\log n)$*.

*Замечание, заметим, что эта амортизационная оценка показывает лишь, 
что при применении большого числа insert без delete в среднем операция будет работать за $O(\log n)$.
Но при чередование операций insert и delete неверно, что в среднем операция будет за $O(\log n)$.
Достаточно взять $n = 2^{k}$, и поочерёдно делать delete и insert, тогда каждая операция будет за $\Theta(n)$,
и значит амортизационное время каждой есть $\Theta(n)$. 
К сожалению, как сделать так, чтобы и в этом случае амортизационное время было $O(\log n)$ не ясно.
\end{document}