\documentclass[12pt]{article}
\usepackage[utf8]{inputenc}
\usepackage[russian]{babel}
\usepackage{soul}
\usepackage{amsmath}
\usepackage{amssymb}
\title{TimSort Теория}
\date{23 октября 2016}
\author{\copyright~~И. И. Гридасов}
 
\begin{document}
\maketitle
Покажем сначала, что первый и третий этап работают за $O(n\log n)$.

\begin{itemize}
	\item Первый этап работает за время $O(n)$. 
		Так как число minRun можно считать константой, то сортировка каждого рана выполняется за $O(1)$.
		Всего Run-ов $O(n)$, поэтому суммарное время $O(n)$.
	\item Третий этап - слияние всех Run-ов в стеке.
		Заметим, что в начале третьего этапа в стеке находится $O(\log n)$ Run-ов.
		Так как в самом верхнем Run-е кол-во эл-тов $\geqslant F_1$, 
		во втором сверху $\geqslant F_2$.
		Далее по индукции получаем, что в $k$-ом сверху Run-е, хотя бы $F_k$ эл-ов.
		Так как в каждом Run-е эл-ов больше, чем сумма двух выше его.
		Но в любом Run-е, кол-во эл-во $\leqslant n$, поэтому $F_k \leqslant n$.
		Но так как $F_k$ можно оценить снизу как $2^{\frac{k}{2}}$, то
		$k \leqslant 2 * \log n$, следовательно кол-во эл-ов в стеке есть $O(\log n)$.
		А слияние двух Run-ов выполняется за $O(n)$, значит суммарная асимптотика есть $O(n\log n)$.
\end{itemize}

Теперь разберёмся со вторым этапом:
Будем в стеке поддерживать следующий инвариант:
для любой тройки подряд идущих Run-ов в стеке с длинами $l_1, l_2, l_3$, записанных от вершины стека,
верно, что $l_2 > l_3$ и $l_1 > l_2 + l_3$.
Будем при добавлении Run-а в стек накидывать на него $P*l*h$ монет, 
где l - длина Run-а, а h - его уровень в стеке.
Рассмотрим операции проталкивания, которые мы совершаем, чтобы 
сохранить инвариант стека при добавлении сверху нового Run-a.
Будем каждый раз рассматривать верхнюю тройку Run-ов и проверять соблюдается ли инвариант,
если нет то как-то сливать 2 Run-а из верхних трёх и проверять инвариант дальше.

Есть два случая, когда инвариант не выполняется:
\begin{description}
	\item[$l_3 \geqslant l_2:$]
		Тогда сливаем Run-ы 2 и 3.
		Пусть второй Run находится на уровне h, тогда 3 на уровне $h+1$.
		Рассмотрим какое кол-во монеток освободилось при совершении данной операции.
		Было: $P*l_3*(h+1) + P*l_2*h$ - стало: $P*(l_2+l_3)*h$, тогда освободилось $P*l_3$ монет.
		Тогда при $P>2$, имеем $P*l_3 > 2*l_3 \geqslant l_2 + l_3$, 
		значит этих монеток хватит на слияние Run-ов 2 и 3.
	\item[$l_3 < l_2$ и $l_1 \leqslant l_2 + l_3:$]
		Тогда сливаем Run-ы 1 и 2.
		Пусть первый Run находится на уровне h, тогда второй на уровне $h+1$.
		Аналогично смотрим на кол-во освободившихся монеток:
		$P*l_1*h + P*l_2*(h+1) - P*(l_2+l_1)*h =  P*l_2$. 
		Тогда при $P\geqslant3$, $P*l_2\geqslant3*l_2>2*l_2+l_3\geqslant l_1+l_3$.
		Значит этих монеток хватит на слияние Run-ов 1 и 2. 
\end{description}

Значит при $P\geqslant3$, нам хватит монеток, чтобы выполнить все добавления Run-ов в стек и 
все слияния, чтобы сохранить инвариант стека.
Осталось понять, что кол-во монеток, которое мы использовали есть $O(n\log n)$. 
Для этого заметим, что в любой момент времени кол-во Run-ов в стеке есть $O(\log n)$(из док-ства в этапе 3).
Тогда при накидывании на Run $P*l*h$ монет, то это тоже самое, что на каждый эл-т Run-а накинуть $P*h$ монет.
Так как $h = O(\log n)$ и $P = 3 = O(1)$, то на каждый эл-т мы накинули $O(\log n)$ монет. 
Значит всего монет поступило $O(n\log n)$ монет. 
Поэтому суммарно второй этап выполняется за $O(n\log n)$.

Наконец, так как все три этапа Алгоритма TimSort выполняются за $O(n\log n)$, 
то и весь алгоритм выполняется за $O(n\log n)$. ч.т.д.
\end{document}